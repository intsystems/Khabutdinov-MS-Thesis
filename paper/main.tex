\documentclass[a4paper,14pt]{extarticle}
\usepackage[utf8]{inputenc}
\usepackage[english,russian]{babel}
\usepackage{makecell}
\usepackage{amsthm}
\usepackage{graphicx}
\usepackage{caption}
\usepackage{amssymb}
\usepackage{amsmath}
\usepackage{mathrsfs}
\usepackage{euscript}
\usepackage{graphicx}
\usepackage{subfig}
\usepackage{caption}
\usepackage{color}
\usepackage{bm}
\usepackage{tabularx}
\usepackage{adjustbox}
\usepackage{multirow}
\usepackage{xcolor}
\usepackage{url}
\usepackage{hyperref}


\usepackage[toc,page]{appendix}

\usepackage{comment}
\usepackage{rotating}

\DeclareMathOperator*{\argmax}{arg\,max}
\DeclareMathOperator*{\argmin}{arg\,min}

\newtheorem{theorem}{Теорема}
\newtheorem{lemma}[theorem]{Лемма}
\newtheorem{definition}{Определение}[section]

\numberwithin{equation}{section}

\newcommand*{\No}{No.}

\begin{document}

% Титульный лист
%\input{./frontpage.tex}

% Аннотация
\input{./annotation.tex}

% Нумерация должна начинаться со второй страницы
\setcounter{page}{2}

% Оглавление
\newpage
\tableofcontents

% Обозначения и сокращения
% \input{./dict.tex}

% Введение
\input{./introduction.tex}


% Введение
\input{./task.tex}

% Введение
\input{./methods.tex}

% Введение
\input{./experiments.tex}

\input{./results_1.tex}

% Введение
\input{./motivation.tex}

\input{./second_task.tex}
\input{./algorithm_description.tex}

\input{./experiments_2.tex}

\input{./results_2.tex}

\input{./conclusion.tex}

% Выводы


% Библиографические ссылки
\input{./bibliography.tex}
% Приложения
%\input{./appendices.tex}

\end{document}
